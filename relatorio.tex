\documentclass[a4paper,12pt]{article}

\usepackage[brazil]{babel}
\usepackage[utf8]{inputenc}
\usepackage[T1]{fontenc}
\usepackage{ae} % Arruma a fonte quando usa o pacote acima

\usepackage{amssymb} % Alguns caracteres matemáticos especiais
\usepackage[top=1in, bottom=1in, left=1in, right=1in]{geometry}
\usepackage[pdftex]{graphicx}%Para inserir figuras



\title{Pinball}

\author{Ronaldo Yang \& Yoshio Mori
  \\
  \small 6432393 \& 6432393
}
\date{Junho de 2014} % Data


\begin{document}
\pagestyle{myheadings}
\maketitle
\newpage
\tableofcontents
\pagebreak

\section{Estrutura}
\subsection{Função Principal (main - pinball.js)}
É a primeira função a ser chamada pelo cliente. Ela carrega o canvas, a interface do webgl, o scene e é reponsável por fazer a integração entre esses sistemas.

A rotina dela consiste em carregar os buffers, inicializar o shader, criar os objetos. Após isso ele entra em loop de atualização dos objetos e renderização.

\subsection{Classe principal (Scene - scene.js)}
Essa classe contém as informações das imagens, de como desenhá-las na tela e das posições, tamanhos e orientações de cada peça.

\subsubsection{Atributos da Classe}
\begin{description}
\item[gl] Interface gráfica.
  \subitem[shaders] Conjunto de Object shader.
\item[images] Conjunto de Object image.
\item[pieces] Conjunto de Object piece.
\item[camera] Conjunto de Object camera.
\item[ready] Boolean informa true se todos os Object image foram carregados.
\end{description}
\subsubsection{Principais Métodos da Classe}
\begin{description}
\item[createShaders] Inicia um novo Object shader obtendo a posição, normal e textura de cada variável no shader.
\item[createImage] Faz a leitura do arquivo Obj, coletando os vértices para serem armazenados num buffer da gpu. Com isso ele instancia um Object image.
\item[isReady] Para cada Object image instanciado, verifica se a imagem já está pronta. Dessa forma é possível determinar quando começar as iterações do jogo.
\end{description}
\subsection{Object image}
Esse objeto contém os endereços dos buffers de posições, normais, textura e indices de cada um dos vértices, assim como o shader reponsável pela renderização.

E seu único método, draw, é chamado pelo Object piece com os parâmetros das matrizes de projeção vértice e modelo. Com esses parâmetros inicia-se o processo de renderização.

\subsection{Object piece}
Cada Object piece está associado à um Object image. Ela contém as informações de posição, tamanho e orientação.

No processo de construção as informações de posição, tamanho e orientação são usados para gerar a matriz de modelo, e sempre que esse atributos são atualizados a rotinha se repete.

\subsubsection{Principais Métodos}
\begin{description}
\item[isReady] Verifica se a imagem foi inicializada.
\item[show] É chamada pela câmera com os parâmetros das matrizes de projeção e visualização. O método envia esses parâmetros para o Object image associado, assim como sua matriz de modelo.
\end{description}
\subsection{Object camera}
Seus atributos são posição, ponto da direção que a câmera está voltado e vetor que indica o cima da camera.

Assim como no object piece, aos ser instanciado as matrizes de projeção e visualização são calculadas e sempre que seus atributos são atualizados novas matrizes são geradas.

Seu método show é chamado pelo scene, que por sua vez mostra a imagem de cada peça.


\section{Objetos}
\mbox{}

Os objetos que estariam na cena estão na pasta pinballOBJ. Todos foram feitos usando o software Blender 2.7.

\section{Eventos do Teclado}
\mbox{}

Os eventos descritos aqui estão no projeto flippers-keyboard.

Para a verificação dos eventos do teclado, assim como a movimentação das palhetas, foi feito um pequeno simulador. Esse, consiste em dois triângulos simulando o que seriam as palhetas e se 
movimentando(rotacionando no eixo z até um certo ângulo e depois voltando a posição original) quando as teclas certas são pressionadas. Para ser mais semelhante as palhetas, os triângulos
deveriam ter sido iniciados transladados um pouco para o eixo x positivo, para que o eixo de rotação fosse mais parecidos com os das palhetas. Como era apenas um teste, não foi dada 
preocupação a isso.

Foi implementado também nesse simulador o botão de start, pause, unpause e restart. A lista de teclas e suas funções são:

\begin{itemize}
\item ENTER para start
\item Z para movimentação do triângulo esquerdo
\item X para movimentação do triângulo direito
\item P para pause
\item U para unpause
\item R para reiniciar
\end{itemize}

Ao iniciar esse simulador, só haverá algum movimento(início) quando se dá start. Ao apertar ENTER, podemos começar a movimentar as palhetas com Z e X. Caso apertamos P em algum momento, 
a cena congela e só continua a sua movimentação assim que apertar U. Para restart, aperta-se R e as palhetas voltam a posição inicial. Para iniciar a movimentação novamente basta apertar 
ENTER.

Outros eventos que foram descritos, mas não implemetados são:

\begin{itemize}
\item SPACE para a força da mola
\item PAGE UP para o aumento da inclinação da mesa
\item PAGE DOWN para a diminuição da inclinação da mesa
\end{itemize}

Para o caso da mola, iríamos achatar a mola conforme o pressionamento da tecla SPACE. Quanto maior pressionamento, maior seria a força na direção da bola. 

\section{Colisões}
\mbox{}

Para a detecção de colisões iríamos fazer o seguinte:

\subsection{Bola-Bola}
\mbox{}

Basta verificar se a distância entre os centros é menor ou igual a soma dos raios.

\subsection{Bola-Poliedro}
\mbox{}

\begin{itemize}
\item Encontramos os limitantes do poliedro nos eixos x, y e z. Assim criamos uma espécie de bounding box, um cubo que envolve/encapsula todo o poliedro
\item Verificamos se existe uma potencial intersecção entre esse cubo e a esfera(bola). Assim evitamos mais cálculos caso não exista colisão, pois é mais
  fácil verificar a intersecção entre uma esfera e um cubo
\item Caso não exista intersecção, não há colisão. Fim
\item Se sim, pode existir uma possível colisão
\item Verificar a colisão através de pontos convenientes do poliedro e da esfera
  
\end{itemize}

\end{document}
